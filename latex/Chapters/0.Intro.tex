\chapter*{Introducción}
\addcontentsline{toc}{chapter}{Introducción}

% La introducción no cuenta como primer capítulo

\noindent Una pregunta que la gran mayoría de personas se ha hecho o le han preguntado es ¿Qué harían si se ganaran la lotería? las loterías alrededor del mundo son muy populares, muchas personas compran boletos de lotería día a día con la esperanza de tener suerte y ganarse algún premio. Es por esto que, el estudio del comportamiento de los individuos y la distribución de ingreso en este tipo de bien ha ido en aumento, analizando diferentes tipos de loterías en diferentes lugares.  \\

En una apuesta, lotería o un juego de azar existen diferentes elementos sobre los cuales depende la probabilidad de ganar. En el caso de una lotería, ganar un premio dependerá de cuantos boletos se van a emitir lo cual se traduce en una probabilidad, casos favorables entre casos totales. A mayor número de boletos emitidos, la probabilidad de ganar un premio si compramos un solo boleto será menor. \\

Podemos aumentar nuestras probabilidades de ganar si compramos más boletos. La venta de boletos de una lotería, como de casi todos los bienes, estará sujeto a la cantidad de ingreso disponible y las preferencias de cada individuo. \\

Para la presente tesis, se trabajó con el Sorteo Tec, institución que realiza diferentes sorteos a lo largo del año. Se obtuvo información de cuatro loterías las cuales tienen diferentes premios, número de boletos emitidos y precios. Estudiaremos si estos boletos son bienes normales al ingreso, lo cual nos dice que a mayor ingreso, aumenta la cantidad de boletos que la gente quiere comprar. Esto tiene suma importancia cuando la economía se enfrenta a una recesión pues, se generan las siguientes preguntas: ¿La venta de boletos caerá? Si la venta cae ¿En qué magnitud será esta caida? Conocer la elasticidad ingreso para los organizadores es de suma importancia para entender qué es lo que le pasará a la institución ante la crisis. \\

La presente investigación se desarrolla de la siguiente manera: en el \textbf{Capítulo 1} se habla sobre los antecedentes de la aversión al riesgo, así como de la literatura expuesta en años recientes sobre la elasticidad ingreso de diferentes loterías. En el \textbf{Capítulo 2} se desarrolla el marco teórico referente a la aversión al riesgo, las loterías y las elasticidades. Se definen conceptos importantes que son necesarios para el desarrollo y análisis de los resultados de la tesis. En el \textbf{Capítulo 3} se realiza un análisis exploratorio de datos con la información proporcionada por el Sorteo Tec para después plantear los modelos econométricos. Se ajustaron dos modelos econométricos, uno para el cálculo de la elasticidad ingreso y otro para encontrar la relación de concavidad de la función de utilidad. El ajuste de estos modelos se realizó a los cuatro sorteos por separado. En el \textbf{Capítulo 4} se analizan los resultados obtenidos en el capítulo anterior. Además, se utilizan las proyecciones del PIB publicadas por diferentes instituciones para estimar el cambio en el número de boletos vendidos de cada uno de los cuatro sorteos bajo estas diferentes situaciones. Finalmente, se presentan las conclusiones del estudio tomando en cuenta toda la información disponible en complemento con los resultados obtenidos.

%Parrafo anterior pero en forma de lista preguntarle a barbara como dejarlo.
%\begin{itemize}
%    \item En el capítulo 1 se habla sobre los antecedentes de la aversión al riesgo, así como de la literatura expuesta en años recientes sobre la elasticidad ingreso de diferentes loterías.
    
%    \item En el capítulo 2 se desarrolla el marco teórico referente a la aversión al riesgo, las loterías y las elasticidades. Se definen conceptos importantes que son necesarios para el desarrollo y análisis de los resultados de esta tesis.
    
%    \item En el capítulo 3 se realizó un análisis exploratorio de datos con la información proporcionada por el Sorteo Tec para después plantear los modelos econométrico. Se ajustaron dos modelos econométricos, uno para el cálculo de la elasticidad ingreso y otro para encontrar la relación de concavidad de la función de utilidad. El ajuste de estos modelos se realizó a los cuatro sorteos por separado.
    
%    \item En el capítulo 4 se analizan los resultados obtenidos del capítulo anterior. Además, se utilizan las estimaciones del PIB realizadas por diferentes instituciones para estimar el cambio en el número de boletos vendidos de los cuatro sorteos bajo estas diferentes situaciones.
    
%    \item Finalmente, se presentan las conclusiones del estudio tomando en cuenta toda la información disponible en complemento con los resultados obtenidos.
    
%\end{itemize}


% Se sugiere que el primer párrafo de cada sección no tenga sangría: \noindent
