\chapter*{Conclusiones}
\addcontentsline{toc}{chapter}{Conclusiones}

% Las conclusiones tampoco cuentan como capítulo

\noindent Desde el primer estudio formal de la aversión al riesgo con Bernoulli en el análisis de las apuestas con la paradoja de San Petersburgo en el siglo XVII, un gran número de estudios se han desarrollado con el paso de los años. Esto con la finalidad de entender el proceso de decisión al que se enfrentan los individuos en la toma de decisiones y el por qué varía tanto entre estos. \\

Se analizaron cuatro diferentes loterías del Sorteo Tec, cada uno con diferentes premios, números de boletos vendidos, precios y por ende diferentes probabilidades de ganar algún premio. A pesar de lo anterior, los resultados obtenidos para los cuatro sorteos fueron muy similares. \\

Para esta tesis, se utilizaron modelos econométricos para la estimación de la elasticidad ingreso y la obtención de la concavidad o convexidad de la función de utilidad. Se realizó un estudio de análisis de sensibilidad de la demanda de boletos de estos diferentes sorteos. Se buscó la relación existente con el ingreso de los individuos para así determinar qué tipo de bienes son. \\

Para los cuatro sorteos que se estudiaron, se obtuvo que son bienes \textbf{superiores al ingreso} ($\varepsilon_{X,I} > 1$). Este resultado es uno de los más importantes pues, al comparar frente a los diversos resultados en la literatura relacionada a la estimación de la elasticidad ingreso de boletos de lotería, la mayoría arrojan que son bienes normales al ingreso ($0 < \varepsilon_{X,I} < 1$). De esta forma, al ser los boletos de los cuatro sorteos bienes superiores al ingreso, el efecto que resulta de un cambio en el ingreso a la cantidad demandada de estos bienes es mayor. \\

En adición a lo anterior, los resultados revelaron que la función de utilidad que representa a los jugadores de estas loterías es cóncava, esto es $U^{\prime \prime} < 0$, aunque esta relación no es tan fuerte debido a que el coeficiente que nos indica esta relación es muy cercana a cero para los cuatro sorteos. De esta manera, podemos concluir que los individuos que participn en estas loterías son \textbf{aversos al riesgo}. Lo que se pierde por comprar el boleto es poco en relación al ingreso inicial con el que cuentan los individuos. Esto implica que los individuos no requieren de una alta probabilidad de ganar un premio para decidir comprar un boleto. Es importante mencionar que el valor esperado de las cuatro loterías es menor al precio de entrada que se paga por particapar en estas apuestas. \\ 

Para los organizadores del Sorteo Tec estos son resultados un tanto ambiguos pues dependerá de la situación en que se encuentre el ingreso de los consumidores. Bajo nuestra hipótesis, en situaciones de crisis económica como la que se está viviendo al momento de escribir esta tesis (recesión, coronavirus, incertidumbre jurídica por parte del Ejecutivo, etc.) la disminución en la venta de boletos no va a ser lineal sino cada vez será mayor. \\

El escenario macroeconómico para el 2020 no figura bien. La economía mexicana ha venido a la baja desde 2019, esto se debe a diferentes factores tanto internos como externos. A lo largo del año diferentes instituciones han publicado sus estimaciones del PIB para 2020, la tendencia a lo largo del año hasta abril del 2020 ha sido recortar estas estimaciones. Bajo los escenarios presentados por estas diferentes instituciones hasta abril, se observa que el impacto en la disminución en el número de boletos vendidos será de una magnitud considerable. Las predicciones muestran disminuciones desde un -3\% hasta un -14\% \\

Con los resultados obtenidos, nuestra hipótesis se sustenta de forma tal que, ante situaciones de crisis económica, el número de boletos vendidos por el Sorteo Tec será menor. El impacto será mayor que la caida en el PIB y esta relación no será lineal. \\

%FALTA ESCRIBIR MAS



Sería interesante para futuras investigaciones realizar un estudio que tome en consideración micro-datos de los participantes de esta lotería para poder tener una mayor descripción de los participantes y ver cómo es la asignación de recursos en cada sector de la población tomando en cuenta variables geográficas, demográficas, ingreso y niveles de educación.

